\tikzstyle{block} = [draw, rectangle, 
    minimum height=1.25em, minimum width=2.5em]
\tikzstyle{sum} = [draw, circle, node distance=1cm]
\tikzstyle{input} = [coordinate]
\tikzstyle{output} = [coordinate]
\tikzstyle{pinstyle} = [pin edge={to-,thin,black}]

% The block diagram code is probably more verbose than necessary
\begin{tikzpicture}[auto, node distance=4cm,>=latex']
    % We start by placing the blocks
    \node [input, name=input] {};
    \node [sum, right of=input] (sum) {};
    \node [block, right of=sum] (g1) {$\frac{-R_6}{(R_5)(1+sC_1R_6)}$};
    \node [block, right of=g1] (g2) {$\frac{-R_8}{(R_7)(1+sC_2R_8)}$};
    \node [output, right of=g2] (output) {};
    \node [block, below of=g2] (feedback_gain) {$\frac{R_{10}}{R_9+R_{10}}$};
    % \node [block, right of=controller, pin={[pinstyle]above:Disturbances},
    %         node distance=3cm] (system) {System};
    % We draw an edge between the controller and system block to 
    % calculate the coordinate u. We need it to place the measurement block. 
    %\draw [->] (controller) -- node[name=u] {$u$} (system);
    % \node [output, right of=system] (output) {};
    % \node [block, below of=u] (measurements) {Measurements};
    % \draw [draw,->] (input) -- node[pos=0.99] {$+$} node {$V_{s}$} (gain);
    % % Once the nodes are placed, connecting them is easy. 
    % % \draw [draw,->] (input) -- node {$r$} ();
    % \draw [->] (gain) -- node {$e$} (sum);
    % \draw [->] (system) -- node [name=y] {$y$}(output);
    % \draw [->] (y) |- (measurements);
    % \draw [->] (measurements) -| node[pos=0.99] {$-$} 
    %     node [near end] {$y_m$} (sum);
    \draw [draw,->] (input) -- node[pos=0.99] {$ $} node {$V_{in}$} (sum);
    \draw [->] (sum) -- node {$ $} (g1);
    \draw [->] (g1) -- node {$  $} (g2);
    \draw [->] (g2) -- node [name=y] {$V_{o}$}(output);
    \draw [->] (y) |- (feedback_gain);
    \draw [->] (feedback_gain) -| node[pos=0.99] {$-$} node [near end] {$V_{f}$} (sum);
\end{tikzpicture}